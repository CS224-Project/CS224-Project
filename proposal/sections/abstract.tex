\section{Abstract}

\vspace{5mm}
\noindent
The United States' Congress is a rich social network where each legislator 
interacts with his peers through committees, house leadership positions, and 
bill co-sponsorships and amendments. We will investigate qualitative measures 
of the network, including the difference in congressional networks using 
different relationship features, how legislators are clustered beyond party 
lines and what features are common in those clusters, and how networks change 
overtime (say for when an incumbent is replaced with a new legislators). 
Further, we will investigate to what extent relationships between legislators 
govern support for the bills the propose and the likelihood of the bill being 
passed into law.

\vspace{5mm}
\noindent
We plan to use data gathered by James Fowler, who did a similar study, on the 
$93^{rd}$ through $110^{th}$ Congress and supplement it with more descriptive 
information on the bills that were proposed and the legislators that proposed 
them. We believe that enriching these datasets will provide us with additional 
opportunities for performing more sophisticated network analysis, such as 
Latent Multi-Group Membership Graph Models, which will allow us to construct 
latent clusters on legislators based on their own features, such as sex, 
party, ideology.