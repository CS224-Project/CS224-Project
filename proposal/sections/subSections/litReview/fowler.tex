\subsection{Fowler}

\vspace{5mm}
\noindent
Fowler presents a graph model for the 93 through 108 Congress of the United 
States. He sought to infer the connectedness between legislators primarily 
through a network of bill co-sponsorships. He claims that legislators 
cosponsoring each others bills is a signal for relationship between legislators.

\vspace{5mm}
\noindent
Specifically, he organized his graph by partitioning by both Congress and house 
(Senate Versus The House of Representatives), which created $32$ 
[$(108 - 93 + 1) \times 2$] distinct partitions. He primarily considered how 
connected legislators are by representing each legislator as a node and a 
directional edge between legislator $A$ to legislator $B$ if legislator $A$ 
cosponsored a bill for legislator $B$. With these partitioned graphs, he 
examines several centrality measures for each graph (i.e. degree, betweenness, 
closeness, and eigenvector) and sees how these measures track through time for 
each individual congressman. He hypothesized that the most connected 
congressmen by centrality members should stay stable over time and evaluated 
this by tracking the most central members and seeing if they remained the 
most central across different congresses.

\vspace{5mm}
\noindent
He continues his study by employing his own weighting metric on a directed 
graph. He hypothesizes congressman $A$ who cosponsors a bill with congressman 
$B$ shows different levels of support for each other depending on the total 
number of individuals who cosponsor a bill - that is, if $A$ is the only 
cosponsor for $B$, this signifies stronger support than if $A$ was one 
cosponsor out of $25$. Fowler constructs a weighted graph such that the weight 
from $A$ to $B$ is for each bill where $A$ cosponsored a bill with $B$, the 
sum of the reciprocal of the total number of cosponsors for that particular 
bill. Using this metric, he infers closeness centrality and clusters between 
congressman and evaluates that the most connected clusters appear to be 
amongst the leadership positions (committee chairs and ranking members)
