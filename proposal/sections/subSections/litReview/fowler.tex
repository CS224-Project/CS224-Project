\subsection{Fowler}

Fowler presents a graph model for the 93 through 108 Congress of the United 
States. He sought to infer the connectedness between legislators primarily 
through a network of bill co-sponsorships. He claims that legislators 
cosponsoring each others bills is a signal for relationship between legislators.

Specifically, he organized his graph by partitioning by both Congress and house 
(Senate Versus The House of Representatives), which created $32$ 
[$(108 - 93 + 1) \times 2$] distinct partitions. He primarily considered how 
connected legislators are by representing each legislator as a node and a 
directional edge between legislator $A$ to legislator $B$ if legislator $A$ 
cosponsored a bill for legislator $B$. With these partitioned graphs, he 
examines several centrality measures for each graph (i.e. degree, betweenness, 
closeness, and eigenvector) and sees how these measures track through time for 
each individual congressman. He hypothesized that the most connected 
congressmen by centrality members should stay stable over time and evaluated 
this by tracking the most central members and seeing if they remained the 
most central across different congresses.

He continues his study by employing his own weighting metric on a directed 
graph. He hypothesizes congressman $A$ who cosponsors a bill with congressman 
$B$ shows different levels of support for each other depending on the total 
number of individuals who cosponsor a bill - that is, if $A$ is the only 
cosponsor for $B$, this signifies stronger support than if $A$ was one 
cosponsor out of $25$. Fowler constructs a weighted graph such that the weight 
from $A$ to $B$ is for each bill where $A$ cosponsored a bill with $B$, the 
sum of the reciprocal of the total number of cosponsors for that particular 
bill. He then infers closeness centrality with these weighted edges (a metric 
he calls connectedness) and clusters between congressman and evaluates that the 
most connected clusters appear to be amongst the leadership positions 
(committee chairs and ranking members). Further Fowler constructs a negative 
binomial model using the centrality and connectedness measures as features to 
predict the number of bills and amendments that are successfully passed by each 
representative. He compares the effect sizes for each model as the ratio of the 
predicted number of amendments passed for a person who had $1$ standard 
deviation greater connectedness or centrality to the  number of amendments 
passed for a person of average connectedness or centrality for each congress 
and notes that connectedness is the metric that is better at measuring a 
difference in effect as compared to all centrality measures. The effects sizes 
range between $39\%$ and $59\%$.

Fowler concludes his study by trying to predict roll-call votes for a bill 
given the connectedness of the sponsor and controlling for ideology measures 
(Poole and Rosenthal 1997). He is able to infer a positive correlation 
indicating that there are more 'aye' votes for a bill when the individual is 
connected, but the effect size is small.