\subsection{Backstrom and Leskovec}

Backstrom and Leskovec propose an algorithm, Supervised Random Walks, designed
to naturally combine node and edge information with network structure in order
to provide superior link predictions and recommendations.

The link prediction and recommendation problems are considered challenging, as
real world graphs are typically very sparse, and it can be difficult to
determine to what extent the network structure is responsible for linking, as
opposed to node and edge features, or even external factors.  To effectively
mingle this structural information and feature data, the Supervised Random Walks
algorithm performs supervised learning steps using the edge features to bias the
random walk over the network so that it will  visit certain nodes more often
than others.  During the learning steps, the  edge features are used to learn
the `strengths' of each edge and assign a  positive or negative value, such that
more positive nodes are visited more  frequently.  In addition, for link
prediction, the positive links are the ones  where edges will be created in the
future, whereas negative links are left  alone.  To assign strengths using edge
features, they trained an algorithm  using a source node and training examples
where edges will be created.

When tested on physics coauthorship and Facebook friendship data, Supervised
Random Walks outperformed both unsupervised and supervised methods.  In
addition, Supervised Random Walks doesn't require feature generation or
extraction, so it is doubly impressive that it beats algorithms that use
supervised machine learning methods with complex feature extraction and
generation schemes to account for node and edge features. Overall the algorithm
appears to offer significant improvements over well known alternatives like
random walks with restarts and supervised machine learning techniques.
