\subsection{Model Evaluation}

\textbf{Cohorts}

Our base model and assumption is that partisanship mostly governs cohorts. 
When evaluating cohorts that we develop through legislator attributes other 
than party, we will compute and evaluate party purity as compared to cohorts 
that are based on party. That is, we will evaluate if a cohort was formed 
due to party alliances by examining the like features of the cohort - if the 
predominately like feature is party, we will assume the cohort formed by party.

\textbf{Bill Passage and Votes}

We will assume that the party of the main bill sponsor is what governs the 
likelihood that a bill will get passed. We will develop a variety of predictive 
models (logistic regression, supervised random walks) that train on features 
other than party on early bill networks. We will examine which features govern 
the likelihood of a bill getting passed. Further, we will try to predict by 
representative their likelihood to vote on a given bill.