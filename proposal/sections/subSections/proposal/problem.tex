\subsection{Problem}

The growth of partisanship in recent years has dominated the political conversation in Washington D.C. The Democratic Party and Republican Party have grown more ideologically distinct and unwilling to compromise. 

We are interested in shedding light on how this evolution has occurred by focusing on the minutiae of governance - the network of sponsorships and cosponsorship of the thousands of bills introduced each two-year cycle, most of which never see the light of day. In particular, we seek to understand the state of working relationships in Congress, how they have evolved over time, and how these evolutions have affected legislative outcomes.

We see a number of direct applications of graph and network analysis that can help us answer these questions. On the qualitative side, we hope to analyze different clusters and coalitions of Congressmembers over time using methods such as the Latent Multi-Group Membership Graph Model discussed above. Which features characterize these coalitions over time; do bipartisan coalitions still exist in any meaningful way? How important are the coalitions to explaining bill outcomes? Who are the power brokers in Congress that have high centrality scores in these graphs; are they senior coalition-builders or party leaders?

There are a number of quantitative models we think would be appropriate as well. To better understand the dynamics of working relationships, we can consider an edge prediction problem: what features best predict whether two Congresspeople are likely to work together? We think the Supervised Page Rank approach is promising. We also think that there might be machine learning approaches to understand why bills pass as a function of their cosponsorship networks - is partisan unity starting to trump early bipartisan support?

Fowler has started down a productive path in thinking about many of these questions, but there is a lot more we can do. On the micro-level, what is the best way to represent the network of Congressional relationships in terms of undirected vs. directed, weighted vs. unweighted, and so on? More broadly, how does partisanship tie into many of the questions he addresses?

We are primarily focused on the state of relationships in congress, how 
they evolved over time, and how those evolutions affected the state of 
legislative outcomes. In particular, we want to investigate what constitutes a 
relationship in congress. Fowler suggested that a relationship is a function 
of who cosponsors bills with a legislator. We believe that this is likely the 
case, but we are interested in using and incorporating other features into 
the graph, such as committee membership. 

We would also like to determine a method for evaluating the Congressional
network as it grows and changes overtime. It is commonly thought that Congress
has become more partisan as time has progressed. We want to see, given our
relationship metric, if we can see partisanship evolving. Further, we want to
see if beyond partisanship if representatives belong to cohorts. We are 
optimistic that the

Fowler attempted to do this by  partitioning each Congress and examine
centrality features to see if they  remained static or not. We would like to try
to model this network in a  large graph

Further, beyond legislative working relationship features we are interested in 
determining the main features beyond partisanship that determines cohorts. We 
would like to see if cohort membership identifies similar bill sponsorship and 
voting behavior.

Partisanship has increased in Congress since the $1970$'s. We are primarily 
interested to see to what extent partisanship has evolved overtime and affected 
Congressional social networks and bill passage.

Fowler suggested that legislative relationships are a function  of who
cosponsors bills with a legislator. We believe that this is likely the  case,
but we are interested in using and incorporating other features into  the graph,
such as committee membership. 
