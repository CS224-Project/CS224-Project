\subsection{Kim and Leskovec}

The Kim and Leskovec paper includes a number of elements that are very relevant to our work. In particular, the focus on node features mirrors with our interest in how much partisanship and ideology affect the workings of Congress. Additionally, the nature of the group memberships is very appealing. It is very natural to assume that Congresspeople belong to multiple overlapping groups instead of being a part of a single cluster. They have their political party, ideological neighbors, committees and subcommittees, geographical neighbors, demographically similar members, and so much more. It is of great interest to understand how important each of these characteristics are and how they have evolved over time.

At the same time, the paper does not perfectly map onto our work. The model is expansive and general with many free parameters that might not be necessary in our context. For example, it is possible to support situations where feature similarity decreases edge likelihood, which is unlikely in the case of cosponsorship networks. Similarly, the model supports core-periphery relationships that might be less applicable in a body with only 435 members; we'd expect less distinction between members. Finally, many formulations of our graph are much denser than the networks studied in the paper. Still, there is a great deal of overlap and it would be interesting to see the results of the latent-group approach.