\subsection{Fowler - Critique}

\vspace{5mm}
\noindent
Fowler's main analysis was to compare his measure of connectedness to 
traditional centrality measures to see if it provides a better predictive 
measure to see if a representative can get an ammendent on a bill passed and 
if the representataive can rally more votes for a bill they cosponsored. He 
did so by constructing a network for each congress and each house separately 
with no links going between them and evaluating his models with congress and 
house. We have a number of issues with this approach:

\begin{itemize}
	\item Segregated networks introduce the memoryless property to the overall 
	social network. It could be that it is assumed that two members in a 
	prior congress will remain connected in a futre congress, but there was no 
	analysis to evaluate this. 
	\item His connectedness metric is based on heuristic parameters with little 
	support beyond intuition. Further, his bucketing strategy for only 
	considering bills within a congress to determine an edge weight is a 
	bias-variance tradeoff consideration that was not evaluated.
	\item His baseline model of connectedness is very noisy, which is something 
	he himself acknowledges (pg 459). His cutoff for whether to connect 
	representatives is if they cosponsored a bill for one another. However, 
	just one cosponsorship does not infer support for an individual. We would 
	assume that consisted cosponsor support would infer a relationship. 
	\item There was no analysis done on distinguishing between a cosponsor and 
	a primary sponsor (someone who owns the bill).
\end{itemize}

\vspace{5mm}
\noindent
Further, Fowler's analysis lacked a descriptive statistics section for his 
measurement of effect size on amendment passage. He mentions that all 
coefficeints are significantly different from $1$ (pg 476), but he never 
mentions at what significance and how he confirmed the significance. It does 
not appear that he validated this model on a future congress.

\vspace{5mm}
\noindent
Overall, Fowler's analysis is too discritized and he constructs models without 
concern for their statistical underpinnings and methods for evaluation. We 
would have liked to see a more temporal evolution of congressional 
relationships. The network grows as a congress continues and some of the 
relationships built in previous congresses clearly carry over to the next. 
Instead of looking at a congress as a distinct $2$ year interval, we want to 
see how the network blossoms over time.