\subsection{Backstrom and Leskovec - Critique}

Link prediction has several possible applications for a network like the one 
we are considering, but on data that is so feature rich it is important to be 
conscious of how the algorithm we use is handling the balance between network 
structure and node/edge features.  The Supervised Random Walks algorithm would
be ideal for this application, as we will be able to account for our node and 
edge features, without requiring costly and time consuming feature engineering. 
In addition to the savings the algorithm provides over a typical supervised
machine learning algorithm, because of the method with which the strength 
function we train in the supervised step is implemented it is compatible with 
any of our combination of categorical and continuous variables, allowing us more
leeway in the subset of features that we select to guide our link prediction.



