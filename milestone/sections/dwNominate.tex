\section{DW-NOMINATE}

Poole and Rosenthal made a major contribution to the quantitative study of political science with their procedure for computing ideological scores of members of Congress. Under the assumption that legislators and bills can be represented as points in two-dimensional ideological space, they solve for an equilibrium which determines these scores in which each legislator probabilistically votes for the bills closer to her.

The resulting DW-NOMINATE scores are useful because ideology is an incredibly important factor in how legislators behave. The polarization between parties is reflected in the widening gap between the DW-NOMINATE scores of Democrats and the scores of Republicans. 

While the model is not posed as a network problem, there is a network-related interpretation: legislators who vote together frequently (e.g. have highly weighted edges between each other) end up with very similar ideology scores. Similarly, those who vote together least would be most ideologically distinct. We can therefore think of DW-NOMINATE as encapsulating the information contained in roll call voting similarity. The task of this paper, then, is to see what additional information can be learned from the information contained in bill cosponsorships.