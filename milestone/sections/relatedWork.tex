\section{Related Work}

\subsection{Fowler}

Fowler~\cite{Fowler} presents a graph model for the $93^{rd}$ through $108^{th}$
Congress of the United  States. He sought to infer the connectedness between
legislators primarily  through a network of bill cosponsorships. He claims that
legislators  cosponsoring each others bills is a signal for a relationship
between legislators. Specifically, he organized his graph by partitioning by
both Congress and house  (Senate versus The House of Representatives), which
created $32$  [$(108 - 93 + 1) \times 2$] distinct partitions. He primarily
considered how  connected legislators are by representing each legislator as a
node with a  directional edge between legislator $A$ to legislator $B$ if
legislator $A$  cosponsored a bill for legislator $B$.

Fowler's main analysis is tracking different centrality measurements overtime.
He also constructs a network using his own metrics and tries to predict bill
passage. We wish to use the data he devised to develop our own network
representing the collaboration of Congress and track how collaboration has
changed through time as Congress has become more partisan.

We believe that Fowler is correct in using the cosponsorships network as a
measure of collaboration between legislators. We, however, think that having
designating a working relationship (i.e. an edge) as whether two legislators
cosponsored each other's bills at least once is too noisy of an indicator. We
develop a less noisy indicator as thresholding on legislators sponsoring at
least a number of each other's bills, trying to target an average edge density
in our networks (see our data and methods section for more details).



