\section{Data and Methodology}

\subsection{Data Overview}

Our primary source of data was aggregated by James Fowler \textbf{CITATION}.
Fowler developed several disjoint of House and Senate social networks for the
$93^{rd}$ through the $110^{th}$ Congress. His data contains the names and
Interuniversity Consortium for Political and Social Research (ICPSR) Ids for all
legislators for each congress, as well as an indicator matrix of which
legislator was a sponsor of a bill and which legislator was a cosponsor of a
bill. With this, we know who sponsored and cosponsored each bills, and thus, 
which legislators each individual Congressman cosponsored a bill with.

We supplement this data with additional information for each legislator, 
including ideology scores provided by DW Nominate \textbf{CITATION}, party, 
age, sex, and region of the United States. Data sources include DW Nominate, 
GovTrack, and \textbf{BLAH}.

\subsection{Constructing the Network}

The questions we are primarily concerned about answering in this paper relate 
to how can we model working and collaborative relationships with legislators. 
We want to achieve this by creating a network between legislators where a 
connection represents a collaborative working relationship. We have a number of 
options to construct the network. We ultimately chose to define a working 
relationship between Congressman $A$ and $B$ if they meet the following 
conditions:

\begin{itemize}
	\item In total there exists at least $5$ cosponsorships between $A$ and $B$ 
	on which either $A$ or $B$ was the primary sponsor.
	\item $A$ needs to cosponsor at least one of $B$'s bills and vice versa.
\end{itemize}

We discuss our choice of this network over others in the following sections.

\textbf{Cosponsorship Versus Role Call Votes}

We can infer that a cosponsorship is almost equivalent to voting for a bill. If 
you are willing to cosponsor a bill then you will most likely vote for the bill. 
We decided against using roll call votes as our measure of collaboration for 
the following reasons:

\begin{itemize}
	\item Cosponsorship contains support signaling. It is a method legislators 
	use to signal to the rest of Congress that they support a bill.
	\item Voting for a bill does not necessarily imply a collaborative working 
	relationship. There are reasons to vote for a bill that goes beyond a 
	working relationship, including voting for a partisan issue or voting for 
	an issue that affects your constituents.
\end{itemize}

In short, we believe that embedded in roll call votes there is a signal of a 
working relationship, but we believe that the signal is obscured by noise 
relating to other voting reasons.

\textbf{Why 5 Mutual Cosponsorships}

We believe that there is a stronger signal in mutual cosponsorships. We believe, 
however, that without some thresholding on the number of cosponsorships, we may 
get too noisy of a signal for working relationships. We think that consistent 
cosponsorships support between legislators is the best way to smooth out and 
detect working relationships. We also added in the requirement that both 
legislators must cosponsor each other's bills. This is because we want to model 
mutual working relationships. Allowing one-way cosponsorships allows for 
relationships where party and committee leaders force the rank and file 
legislators to cosponsor their bills.

We constructed networks with varying levels of mutual cosponship thresholds and
examined their edge densities. We wanted had two goals when searching for a
threshold. We wanted to hone in on a reasonable average number of working
relationships proportional to the size of the network, and we wanted the density
to be on the same magnitude across all congresses. For the House, we thought 
that on average, having about $20$ working relationships (relative to the usual 
$435$ members of the house) would be ideal. This would corresponds to an average 
density of approximately $0.05$. For the senate, this would be about $0.1$ 
average density.

From the plots of density, we see that we achieve the desired average density 
with minimal spread between congresses for the house at \textbf{K = BLAH} and 
for the senate at \textbf{K = BLAH}.

\textbf{Final Networks}

We implement this methodology across all congresses. Since the Senate and the 
House cannot cosponsor each others bills, we construct disjoint graphs for 
the Senate and House. Also, since we are looking at the evolution of communities 
over time, we construct these networks separately for each congress.

\subsection{Data Manipulations}

We manually changed some of the Fowler unique identifiers to match those in  DW
Nominate and GovTrack. Most of these changes occured when a Legislator  changes
party. For a detailed list of changes, please refer to our github repository.
