\section{CESNA Ideology}

An interesting question is how ideologically diverse the clusters of working relationships are and how this has changed over time. Is it true that there used to be more diverse clusters in the past but that in an era of high polarization we now find only like-minded clusters? The metric we use to answer this question is the average of the within-cluster ideological variances in each Congress.

\begin{figure}[htbp]
  \centering
  \begin{minipage}[h]{0.4\textwidth}
    \includegraphics[width=\textwidth]{charts/cesna_ideo_dim1.png}
  \end{minipage}
  \hfill
  \begin{minipage}[h]{0.4\textwidth}
    \includegraphics[width=\textwidth]{charts/cesna_ideo_dim2.png}
  \end{minipage}
\end{figure}

The evidence does not convincingly support the view of increasing homogeneity. There is some slight evidence that along the second ideological dimension (social issues), working-group clusters have become more similar. On the primary dimension, however, there does not appear to be much of a trend. The 104th Congress, which has shown up a few times throughout our analysis, exhibits abnormally low heterogeneity in the Senate but levels return to normal thereafter. 