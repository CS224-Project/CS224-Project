\subsection{Community Sensitivity and Stability}

Currently our analysis is done without community sensitivity. We allow CESNA to
pick the number of communities and we performed our preliminary analysis on
these communities. In our final report we plan to do more extensive testing on
the stability of the communities that we develop.

The CESNA algorithm and community assignment is dependent on several things: the
number of total communities that we wish to create, the edges distribution in
the graph, and the attribute values within the graph. When we analyze community 
stability, we will perturb each individual and jointly.

\textbf{Edge Perturbations}

For each community in a given Congress, we will detect how robust it is to edge 
deletions. That is, for each edge within a community, we will randomly delete 
them it with varying probability and rerun CESNA to see how the resulting 
communities compare to the original. We will use Jaccard similarity between 
communities to define how similar communities are. A robust community will be 
one that maintains a high Jaccard similarity with reasonably low probability of 
random edge deletion. Communities that completely dissolve with small edge 
probability of edge deletions will be deemed Unstable.

\textbf{Feature Perturbations}

CESNA Only takes in discrete features. However, some of the underlying features 
(e.g. age and ideology) are really discretized continuous variables. As such, 
our perturbation strategy for variable perturbation depends on the type of 
variable.

For underlying continuous variables, we will add random normal noise to the
variables. The average noise will be proportional to that feature's sample
average. We will vary the standard deviation for sensitivity purposes. After
perturbations, we will measure the community's stability by computing the
Jaccard similarity for each community. Again, communities with low Jaccard
similarities will be considered unstable, and we will then reduce the number of
communities that we detect.

For genuinely discrete variables (such as sex), we will employ a strategy
similar to edge perturbations. Except, instead of randomly deleting edges, we
will randomly switch features for all individuals within the cluster. We will
use Jaccard similarity to gage how stable a community is, imputing that low 
Jaccard similarity means an unstable community, thus necessitating lowering 
the number of communities that we use.
