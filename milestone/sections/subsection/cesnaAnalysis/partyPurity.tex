\subsection{Party Purity Analysis}

This analysis tries to estimate how the community's purity in relation to party
has changed over time. To compute purity for a binary feature, we first
determine the proportion of members in a community that exhibit that feature. We
will call this proportion $p$. The purity of a community is then defined as $p
\times (1 - p)$. Note that $p$ ranges from $0$ to $1$. A value of $p = 1$ or $p
= 0$ corresponds to a community comprised of only one type of feature, and
results in a purity of $0$. We note that the largest the purity criterion can be
is when $p = 0.5$, then purity equals $0.25$. Then, our community is very
`impure`.

We proceed with this analysis by first computing the purity of of each community
in each congress. We then compute the average purity between each community
within each congress. We then examine the trend of average community party
purity as it differs from congress to congress.

We hypothesize that working relationships in Congress have become more strenuous
over time as Congress has become more partisan. If this were true, then we would
expect to see a decrease over time in community party purity as Republicans and
Democrats mutually refuse to work with the other party.

\begin{figure}[h!]
    \includegraphics[scale=0.4]{charts/avePurity.png}
\end{figure}

We notice first that the senate does not appear to have an overall trend up or
down. It does, however, have a very sharp decline for the 104 Congress. Recall
that this was the Congress that saw the very sharp decline in average degree.

The House plot, on the other hand, appears to have a very mild decline in
average  purity. Given that we have not finalized our communities or performed
perturbation analysis, and the fact that the decline is so mild, we would not
argue that this is evidence of a more partisan House. We will re-plot this after 
we obtain definitively stable communities.