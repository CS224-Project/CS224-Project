\section{Abstract}

The United States Congress is a rich social network where each legislator
interacts with their peers through committees, house leadership positions, and
bill co-sponsorships and amendments. Through the lens of growing partisanship,
we investigate qualitative measures of the network, including the difference in
congressional networks using different relationship features, how legislators
are clustered beyond party lines and what features are common in those clusters.
We investiage how these clusters and communities evolve over time through
different congresses and evaluate how partisianship has affected these
communities and working relationships in Congress.

We achieved this by first building a cosponsorship network of the $93^{rd}$
through $110^{th}$ Congress and supplemented the network with information about
each Congressman (node attributes). We then implemented Community Detection in
Networks with Node Attributes (CESNA) on the network and tracked communities and
community attributes over time, noting how important partisianship was in
defining each community.

Further, we investigate to what extent early features within the same networks
can predict the success of a bill being passed into law.
