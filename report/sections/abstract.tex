\section{Abstract}

Growing partisan polarization has dominated the conversation about Congress in
recent years. While there have always been issue differences between the major
parties, these differences have grown and calcified along multiple
dimensions~\cite{Layman}. Quantitative metrics based on roll-call votes that
label legislators as points in multi-dimensional ideological space have borne
out the story of major polarization~\cite{Poole}.

\begin{figure}[htbp]
  \centering
  \begin{minipage}[h]{0.4\textwidth}
    \includegraphics[width=\textwidth]{charts/dw_polarize_93.png}
  \end{minipage}
  \hfill
  \begin{minipage}[h]{0.4\textwidth}
    \includegraphics[width=\textwidth]{charts/dw_polarize_110.png}
  \end{minipage}
\end{figure}

At the same time, the United States Congress is a rich social network where each
legislator interacts with their peers through committees, house leadership
positions, and bill co-sponsorships and amendments. The extent to which these
relationships have changed in recent years is not as well understood as the
changes in ideology. If ideologically distinct groups are still able to work
together on crafting bills, and it turns out that those working relationships
are important in legislative outcomes beyond the ideological content of bills,
then partisan polarization in roll-call voting might be less worrying than it
appears.

We build out a network using bill sponsorship-cosponsorship relationships to
track the state of working relationships over time and the nature of legislative
deal-making. Legislators may vote together relatively infrequently but still
find some areas of common ground and work together on bills in those areas. They
may also form communities of multiple legislators and influence which bills get
taken up by committee and ultimately succeed. These and many more features of
the working relationship network can give more fine-grained insight into
Congressional activity, and how it has changed over time, than just the final
votes.

In particular, we build a cosponsorship network of the $93^{rd}$ through
$109^{th}$ Congress and supplement the network with information about each
Congressman (node attributes). From this we are able to track broad patterns of
how Congress has changed over time such as how many and and what kind of working
relationships exist in Congress.

Next, we build stable clusters in the network that represent real-life
communities of legislators based on their bill-writing activity. To do this, we
implement the Community Detection in Networks with Node Attributes (CESNA)
algorithm on the network and track communities and community attributes over
time, noting how important partisianship was in defining each community.

Finally, we investigate to what extent the sponsor and early cosponsors of a
bill can provide insight on the chances of a bill succeeding. Here, we draw upon
not just various features of the working relationship network but also data from
multiple sources characterizing the demographics and prominence of different
legislators. We are interested not just in how predictable bill outcomes are but
on what factors matter over time.
