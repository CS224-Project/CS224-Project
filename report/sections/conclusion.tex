\section{Conclusion}

Party and ideology dominate every discussion about Congressional activity and
rightly so. Yet even as partisanship has grown to record levels, a look at the
network of working relationships reveals that things are not quite as polarized
as a simple look at roll call votes would reveal. Bipartisan working
relationships are as plentiful as ever and nonpartisan variables like region and
committee remain instrumental in characterizing the clusters of working groups
that form in the House and Senate. Within the clusters that form, ideological
heterogeneity has if anything been increasing which is the opposite of what one
would expect.

When it comes to bill passage, ideology is still the most important factor in
determining legislative outcomes. But ideology alone, as represented by the
ideology of the sponsor and early cosponsors, does not tell the whole story of a
bill. Congressmembers with more numerous and important working relationships are
significantly more successful in getting their bills passed. Strong
relationships to multiple factions within one's party and across the aisle, and
the flexibility and compromise that having those relationships signifies, are
critical to achieving legislative success.

Going forward, it is important that we monitor not just the ideological gap
between the parties but also the state of collaboration in general. If the web
of working relationships, which has thus far been managing to stay healthy,
starts to deteriorate, the level of Congressional dysfunction will only
continue to grow.
