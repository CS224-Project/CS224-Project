\section{Bill Passage}

The web of working relationships in Congress is also interesting because of the
effect it may have on legislative outcomes. A basic model of legislation, along
the lines of the Median Voter Theorem~\cite{Black}, would say that legislation
is most likely to pass if its ideological content matches most closely the
median ideology of legislators. A more sophisticated take would include the
effects of party such as the level of support by the majority party and party
leaders, who can exert control on which amendments or bills will come up for a
vote. Under this view of legislative activity, partisan polarization is a
serious and potentially fatal roadblock to legislative compromise and
productivity.

But there are several reasons to believe that there are factors beyond just
party and ideology that would determine the success of a piece of legislation.
For one, relatively few bills are passed relative the the amount introduced, so
there is the question of which bills will be prioritized. Additionally, bills
are often technical in nature without a major ideological component or are
compromises between various factions. Finally, bills themselves might be
subjects of larger compromises, in which legislators vote for each others' bills
in what is known as `logrolling'~\cite{Schwartz}.

Each of these factors can be influenced by considering the network of working
relationships. Congresspeople with more numerous, important, and strategic
relationships may be more likely to get the bills they write or cosponsor
passed, even holding fixed party and ideology variables. If working
relationships remain important to the success of a bill, beyond its ideological
content and even as the parties have moved further apart, that suggests that
there is still a role for compromise and negotiation.

In many ways, the question of what makes bill passage more likely is similar to
the question of community growth in the network literature. The success of a
bill is analagous to the spreading of the early community formed by the early
cosponsors to cover a majority of Congress. Backstrom et al.~\cite{Backstrom}
identify several features of early networks that they see as important for
growth. In particular, they find a lot of significance in the number of nodes
that have edges to someone in the early network, which they call the `fringe'.
Additionally, they look at features related to clustering in the early network,
such as the ratio of closed to open triads which is negatively correlated with
growth. Finally, they look at the `activity' of the initial group, which in our
context might include the legislative productivity of the early cosponsor
networks.

\subsection{Features}

We split our feature set into \textit{demographic} features, \textit{visibility}
features, and \textit{network} features over the early cosponsorship network. In
the case where a bill does not have any cosponsors, the features are computed
for the sponsor alone. To focus on the main backers and supporters of a bill and
to avoid including cosponsors who join onto a bill when it looks like it will
pass, we limit the pool of considered cosponsors to only those who sign on in
the first 30 days after the bill was introduced.

The demographic features most prominently include party and ideology, and
include age, gender, and region as well. These features are meant to capture the
palatability of the bill to various factions. If a bill's chance of success
really does depend primarily on its content, we would expect these demographic
variables to be the important ones. In particular, we consider
\begin{itemize}
	\item whether sponsor is from majority party
	\item party impurity of early cosponsors
	\item ideology of sponsor
	\item mean ideology of early cosponsors
	\item variance of ideology of early cosponsors
\end{itemize}

The visibility features are meant to capture how prominent or well-known a
legislator is in Congress. In some ways, these can be thought of as a poor man's
version of the network features, which explicitly capture how embedded a
legislator is in the Congressional network. The idea is that more visible
legislators will have greater success in attracting support for their
initiatives. All variables that are computed in `real time' are measured as of
the first of the month after the bill was sponsored to reflect the state of the
world after the initial cosponsorship network was formed. Our specific
considered features are
\begin{itemize}
	\item number of terms served by sponsor
	\item whether the sponsor is a chairman or ranking member of a standing committee
	\item number of bills written by sponsor so far
	\item number of bills cosponsored by sponsor so far
	\item number of consponsorships received by sponsor so far
\end{itemize}

Finally, we introduce our network features gleaned directly from the constructed
web of working relationships. We construct the network in `real time' for each
bill, seeding it with the network from the prior Congress and keeping it up to
date as of the bill's date so as to capture recent relationships and the
situation for new Congressmembers. We view these relationships as the
operationalization of the theory mentioned above that more visible legislators
will succeed more often with their bills. Legislators with more useful working
relationships with their peers may be more likely to get their bills taken
seriously as good-faith legislative efforts, prioritized, and passed. It is
useful to have more working relationships but it is also useful if those working
relationships are with other influential members of Congress. Furthermore, it
may be better if the early cosponsorship network has a diverse set of working
relationships and don't cluster too tightly among themselves. These intuitions
lead us to the following set of features
\begin{itemize}
	\item degree of sponsor
	\item eigenvector centrality of sponsor
	\item cut size of set of early cosponsors
	\item clustering of early cosponsors
	\item presence of early cosponsors in stable CESNA clusters
\end{itemize}

\subsection{Variable Importance and Significance}

\textbf{Joint Variable Significance}

To demonstrate that certain sets of variables were significant, we first
developed a logistic regression model using all of our data across all
congressional years, separately for the house and senate. We then used an F-test
for each regime of variables outlined in our feature section (ie. visibility
features, network features, demographic features) to see if in conjunction they
are all significant predictors or not. The results of the F-tests are as follows:

\begin{table}
\centering
   \begin{tabular}{llcc}
   \toprule
   Chamber & Vars & F-Statistic & p-value \\ \midrule
   House & Demographic & 37.27 & 0.0 \\
   House & Visibility & 150.93 & 0.0 \\
   House & Network & 158.84 & 0.0 \\
   Senate & Demographic & 54.93 & 0.0 \\
   Senate & Visibility & 281.04 & 0.0 \\ 
   Senate & Network & 35.89 & 0.0 \\ \bottomrule
   \end{tabular}
   \caption{F-Tests of variable significance on different subsets of data}
   \label{table:perf}
\end{table}

As we can see, each regime of features is highly significant in our models. This
is evidence that each regime is providing joint importance in our model.

\textbf{Variable Importance}

We note from the prior subsection that most of the network features are some of
the most important features in the each logistic regression model, followed by
ideology. However, we are unsure if we can trust these results due to the
liklihood of strong colinearity between our features (for example, our network
features are likely colinear since they approximately measure similar network
features) and strong interaction effects. As such, we investigated a model that
is robust to colinearity and can model interaction effects. We chose to use a
boosted tree model.

We fit a boosted tree model separately for each chamber and year of congress. We
examined the variable importance for each model and plotted the rank of each
feature's importance as it changes over time. We wished to see if certain
features became more or less important over time.

By far the most consistently important features were our ideology features,
followed by network centrality measures and presence in CESNA communities. We
show plots for ideology dimension 1 and eigen centrality, omitting other
ideology and network measures. These other measures behave similarly to these
two.

\includegraphics[width=0.4\textwidth]{charts/idImport.png}

\includegraphics[width=0.4\textwidth]{charts/EigImport.png}

\includegraphics[width=0.4\textwidth]{charts/cesnaImport.png}

Notice that these features are more or less stable in terms of importance over
the years. This implies that those features that were important during the
$94^{th}$ congress remain important even to the $109^{th}$ congress. This story
appears to be the same for most features. Most features rarely change importance
significantly.

\subsection{Predictive Performance}

Predictive models in both the House and the Senate bore out the finding above that network features are important to predicting bill success. Interestingly, crude approximations of working relationships like the number of cosponsorships given out and the number of cosponsorships received (contained in the `visibility' section) did not meaningfully add predictive power. It was only when considering variables related to our network of actual working relationships that a boost was received above and beyond ideology and demographic variables alone.

\begin{table}
\centering
   \begin{tabular}{llcc}
   \toprule
   Chamber & Vars & F1 Score & Log Loss \\ \midrule
   House & Demographic & 0.062 & 0.487 \\
   House & D + Visibility & 0.063 & 0.480 \\
   House & D + V + Network & 0.090 & 0.470 \\
   Senate & Demographic & 0.585 & 0.524 \\
   Senate & D + Visibility & 0.573 & 0.523 \\ 
   Senate & D + V + Network & 0.627 & 0.512 \\ \bottomrule
   \end{tabular}
   \caption{Model performance on different subsets of data}
   \label{table:perf}
\end{table}

The F1 score, a weighted average of precision and recall, depends on the actual 0-1 predictions. As a result, it is more jumpy. Adding visibility variables makes essentially no impact as few predictions flipped, even if predicted probabilities changed. Meanwhile, adding the network moved enough predicted probabilities across the 50\% threshold, leading to sharply improved F1. On the other hand, log loss, which steadily rewards more accurate probabilistic predictions, shows smaller but still substantial gains to adding network features.
