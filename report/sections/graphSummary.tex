\section{Graph Summary Stats}

We start by visualizing some of the working relationship graphs. The two
examples we choose are from the Senate (which has fewer nodes) and depict a
high-density Congressional session (101st) and a low-density session (104th).

\begin{figure}[htbp]
  \centering
  \begin{minipage}[h]{0.25\textwidth}
    \includegraphics[width=\textwidth]{charts/sen_l_color.png}
    \caption*{104 Congress - Senate Network}
  \end{minipage}
  \hfill
  \begin{minipage}[h]{0.25\textwidth}
    \includegraphics[width=\textwidth]{charts/sen_h_color.png}
    \caption*{101 Congress - Senate Network}
  \end{minipage}
\end{figure}

One thing that is immediately noticeable is the party clustering. There is a
core of party loyalists in each graph who are densely connected to one another.
Nestled among the party loyalists are a couple members from the other party.
Then there are members who work closely with members from both parties who have
less dense subnetworks but many edges to disparate parts of the graph.

There are also noticeable authority figures in both parties who have strong
working relationships with numerous others. On the other end of the spectrum are
a handful of Senators with no strong working relationships at all; these members
appear to be cashing their paycheck and not doing much legislating.

The two graphs side-by-side also illustrate just how much things can change
between Congressional sessions. The definition of an edge is the same in both
graphs but the number of edges created is starkly different between the two
Congressional sessions depicted. The brutal and combative 104th Congress,
culminating in a government shutdown, saw substantially less legislative work
done and working relationships developed then the 101st just a handful of years
earlier.

\includegraphics[width=0.4\textwidth]{charts/avgdegree.png}

Going along with this last observation, we can visualize the average degree of
the working relationship graph across time. Changes in party control are labeled
as dots on the chart. After the first few Congresses, which as explained had
different rules for cosponsorships, we see a burst of legislative activity and
relationships between the 98th and 102nd Congress under Ronald Reagan and George
H.W. Bush before a decline and recovery late in the Bill Clinton years and under
George W. Bush.

\begin{figure}[htbp]
  \centering
  \includegraphics[width=0.4\textwidth]{charts/house_deg_dist.png}
\end{figure}

Building on the point about party authorities, meanwhile, we see a degree
distribution that like most networks has heavy tails; the Senate's plot looks
quite similar to that of those House. Most members have only a handful of close
working relationships but there is a long tail of legislators who work
extensively with a substantial fraction of the entire body. Understanding the
work of these high-degree legislators is important to understanding the work of
Congress as a whole.

\vspace{3mm}

\noindent
\textbf{Consistent Bipartisanship}

\vspace{3mm}

\includegraphics[width=0.4\textwidth]{charts/pct_bipartisan.png}

We can also look at the percent of edges over time that cross party lines.
Interestingly, this plot is roughly stable over time, with a possible slight
decline in general from beginning to end keyed by the famous 104th Congress.
This is nothing like the dramatic growth in party polarization demonstrated by
the DW-NOMINATE scores; Congresspeople are forming working relationships across
the aisle in similar proportions to when ideological disparities between parties
were much greater. As a methodological note, we found that the same pattern held
even when controlling for the change in possible bipartisan edges due to
differently sized majority and minority caucuses.
