\section{Data and Methodology}

\subsection{Data Overview}

Our primary source of data was aggregated by James Fowler~\cite{Fowler}. Fowler
developed several disjoint of House and Senate social networks for the $93^{rd}$
through the $110^{th}$ Congress. His data contains the names and Interuniversity
Consortium for Political and Social Research (ICPSR) Ids for all legislators for
each congress, as well as the list of all bills, amendments, and resolutions
instroduced and whether they passed. Importantly, he also has an indicator
matrix of which legislator was a sponsor of a bill and which legislator was a
cosponsor of a bill. With this, we know who sponsored and cosponsored each
bills, and thus,  which legislators each individual Congressman cosponsored a
bill with.

We supplement this data with additional information for each legislator. We get
party, region, and most significantly ideology scores from DW Nominate~\cite{DW-
NOMINATE}. We get other demographic variables like age and sex from GovTrack,
another major database of Congressional information~\cite{GovTrack}. Finally, we
get information on committee membership and seniority gathered from the
Congressional Record and CQ Press by Garrison Nelson, Charles Stewart, and
Jonathan Woon~\cite{Nelson, Stewart}. To our knowledge, we are the first to
combine this many sources of information about Congresspeople and their working
relationships.

The data from GovTrack had many missing identifiers for the $110^{th}$ Congress,
which made joining data for this Congress impossible. Because of this, we
omitted analysis for this congress.

\subsection{Constructing the Network}

The questions we are primarily concerned about answering in this paper relate 
to how can we model working and collaborative relationships among legislators. 
We want to achieve this by creating a network between legislators where a 
connection represents a collaborative working relationship. We have a number of 
options to construct the network. We ultimately chose to define a working 
relationship between Congressman $A$ and $B$ if they meet the following 
conditions:

\begin{itemize}
	\item For the House, there exists at least $4$ cosponsorships between $A$ and $B$ 
	on which either $A$ or $B$ was the primary sponsor. For the Senate, there 
	needs to be at least $12$.
	\item $A$ needs to cosponsor at least one of $B$'s bills and vice versa.
\end{itemize}

We discuss our choice of this network over others in the following sections.

\vspace{3mm}

\noindent
\textbf{Why Not Mutual Cosponsorship}

\vspace{3mm}

Instead of considering only cosponsorships of one-another's bills, one might
consider instead building a network that accounted for mutual cosponsorships of
a third legislators' bills. It's certainly true that at times, cosponsoring a
bill together might mean working together on shepherding the bill through
Congress. However, there are many other times when a simple mutual cosponsorship
might be more of a shared ideological expression than anything else and not
imply any sort of actual working relationship between the mutual cosponsors.
Since we are already capturing ideological affinity with DW-NOMINATE scores, we
don't want our collaborative network to cover the same ground. Cosponsoring
someone else's bill and having them cosponsor yours, on the other hand, requires
direct interaction and a public signal of a willingness to work together.

\vspace{3mm}

\noindent
\textbf{Determining the Cosponsorship Threshold}

\vspace{3mm}

Not all bills, resolutions, or amendments are serious affairs that require a
great deal of collaboration. Some cover trivial matters like commemorations that
are cheap opportunities to signal support for a popular cause via a
cosponsorship. As a result, absent a cosponsorship threshold, we will get too
noisy of a signal for working relationships. Requiring consistent cosponsorship
support between legislators is the best way to smooth the noise out and  detect
bona fide working relationships. We also add in the requirement that both
legislators must cosponsor each other's bills. This is because we want to model
mutual working relationships. One-way cosponsorships could include any number of
situations that are not working relationships such as party pressure to
cosponsor the bills of leadership or the aforementioned cheap cosponsrships of
popular resolutions in which you are one of a hundred cosponsors.

To find the specific threshold, we constructed networks with varying levels of
mutual cosponship thresholds and examined their edge densities. We had two goals
when searching for a threshold. We wanted to hone in on a reasonable average
number of working relationships proportional to the size of the network (5-10\%,
probably with the Senate on the higher end of that range to avoid having too few
working relationships in absolute terms), and we wanted the resulting edge
density to be similar for each Congress's network.

\begin{figure}[h!]
    \includegraphics[scale=0.25]{charts/housedensity1.png}
\end{figure}

\begin{figure}[h!]
    \includegraphics[scale=0.25]{charts/senatedensity1.png}
\end{figure}

From the plots of density, we see that the average density of the House dips
much more sharply (and gets smaller gaps between Congresses) than that of the
Senate as a function of the threshold. Choosing a threshold of 4 for the House
and 12 for the Senate works well in balancing our goals: about 5\% density for
the House (or 20 working relationships on average) and 10\% density in the
Senate (or 10 working relationships on average), both of which are reasonable,
while seeing relatively low variation between Congresses. The higher density on
the Senate side is further justified by the high threshold needed just to reach
that limit: 12 cases of working together is certainly significant.

\subsection{Data Manipulations}

We manually massage the legislator IDs in Fowler, DW- Nominate, Stewart, and
GovTrack in order to get them to agree with one another and give legislators
unique IDs. Most of these changes occured when a Legislator changes party. For a
detailed list of changes, please refer to our github repository.

\subsection{Change In Law}

Starting with the $98^{th}$ Congress, a law passed which allowed a bill to have
more than $25$ legislators cosponsor it at a time. As a result, legislators
started to cosponsor more bills than usual. We can see this behavior manifest
itself in the uptick in average network degrees in congresses after the
$97^{th}$. As such, the behavior and the number of communities we receive from
before the $98^{th}$ congress appear to be fundamentally different.
