\subsection{CESNA Clustering Description}

We are interested in analyzing clusters of legislators based on their network of
working relationships. Crucially, we are particularly interested in how these
clusters relate to the attributes of the legislators. We have seen that there
remain a significant number of individual bipartisan working relationships but
do we see diverse working groups that frequently collaborate together?

A promising approach is laid out in Yang et al.~\cite{Yang}, who describe an
algorithm for finding Communities from Edge Structure and Node Attributes, which
they call CESNA. CESNA incorporates node features and edge structure to find
communities instead of just relying on one or the other. It explicitly computes
the importance of different node features in forming each cluster. Additionally,
it allows for overlapping and nested communities, which is one of the key
features of our dataset. Legislators are likely to have distinct communities of
working relationships with their regional peers, ideological peers, committee
peers, and of course party peers.

In particular, we make use of the authors' C++ implementation and feed in
details of our own network. As is necessary for the models, we binarize all of
our variables. This means that categorical variables like region are split into
distinct dummy variables and continuous variables like ideology are split into
buckets. We choose to split each ideological dimension into five groups: left,
center-left, moderate, center-right, and right, each of which roughly contain a
fifth of the members. The motivation behind selecting an odd number of groups is
to allow for a moderate group that straddles the DW-NOMINATE center point of 0.
The final list of variables, appropriately binarized, are party, ideology,
region, committee, gender, and age.
