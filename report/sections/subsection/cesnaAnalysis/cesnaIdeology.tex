\subsection{Ideological Heterogeneity}

An interesting question is how ideologically diverse the clusters of working
relationships are and how this has changed over time. Is it true that there used
to be more diverse clusters in the past but that in an era of high polarization
we now find only like-minded clusters? The metric we use to answer this question
is the average of the within-cluster ideological variances in each Congress.

\begin{figure}[htbp]
  \centering
  \begin{minipage}[h]{0.4\textwidth}
    \includegraphics[width=\textwidth]{charts/cesna_ideo_dim1.png}
  \end{minipage}
  \hfill
  \begin{minipage}[h]{0.4\textwidth}
    \includegraphics[width=\textwidth]{charts/cesna_ideo_dim2.png}
  \end{minipage}
\end{figure}

The evidence does not support the view of increasing homogeneity. If anything,
it is fairly clear that along the primary ideological dimension (economic
issues), working-group clusters in the House have become more diverse, while on
the second dimension heterogeneity has remained stable. The Senate sees higher
variance from year-to-year (punctuated by the 103rd Senate which is missing
stable clusters altogether) but broadly follows the patterns of the House.

It is again striking how different this is from the sharp ideological clustering
demonstrated in roll-call voting. When it comes to groups of legislators working
together on bills, there is as much ideological diversity today, if not more, as
there has been in the last 40 years.
