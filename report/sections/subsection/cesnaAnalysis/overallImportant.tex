\subsection{Overall Important Features}

\includegraphics[width=0.4\textwidth]{charts/cm_avg_pic.png}

As expected, party and ideology are among the most important features when deciding working group clusters. But there are a fair number of interesting insights here as well. First, ideology is more likely to be an important contributor to a working group than party. That suggests that legislators often gather with like-minded colleagues who may not be homogenous from a party-standpoint. Second, region proves to be an extremely strong contributor to working groups even beyond party in the Senate. It's natural for legislators to gather with others who confront similar regional issues even when that involves working across the aisle. Third, committee proves to be very important in the Senate, again trumping partisanship. This reflects the traditional wisdom that Senate committees are deliberative bodies working through complex legislation while legislative activity in the House may be more governed by partisanship and ideology. Fourth, there is little evidence of legislators forming working groups based on the social issues (2nd) dimension of ideology; economic issues (1st dimension) predominate.