\subsection{Party Impurity}

One powerful application of CESNA clusters is to analyze how bipartisan working
group clusters have been over time. To track this, we use the metric of party
impurity. The party impurity of a cluster is defined as $p \times (1-p)$, where
$p$ is the proportion of the cluster belonging to the Democratic Party (or
equivalently, the Republican Party). Note that $p$ ranges from $0$ to $1$. A
value of $p = 1$ or $p = 0$ corresponds to a community comprised of only one
party, and results in an impurity of $0$. We note that the largest the impurity
criterion can be is when $p = 0.5$; then, impurity equals $0.25$.

We proceed with this analysis by first computing the impurity of of each stable
community in each congress. We then compute the average cluter impurity within
each congress. We then examine the trend of average community impurity as it
differs from congress to congress.

\begin{figure}[h!]
    \includegraphics[scale=0.4]{charts/avePurity.png}
\end{figure}

The results conclusively show that collaborative communities have not become
more partisan over time. The Senate in particular saw dips in bipartisanship
during the 103rd and 104th Congress but that is within a context of higher
variance in general. In the House, things are more stable; working group
clusters have had an impurity of around 0.08 for the past 40 years.
